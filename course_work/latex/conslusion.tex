В рамках курсовой работы была подготовлена база изображений земной поверхности,
были проведены эксперименты по использованию различных подходов к задачам
классификации и сегментации изображений, сделаны выводы о применимости каждого
из них. Была выявлена проблема различия данных разных поставщиков, и предложены
способы её частичного решения. Также была исследована возможность применения
данных OpenStreetMap для сегментации. Выяснилось, что они подходят для детекции
зданий, но не до сегментации. Открытой темой для исследования остаётся
использование алгоритмов доменной адаптации для решения проблемы вариативности
данных, попробовать использовать мультиспектральные данные, находящиеся в
открытом доступе, разработать алгоритм, устойчивый к искажениям данных.
