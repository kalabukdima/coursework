Сегментация изображения --- это разделение изображения на области, однородные по
некоторому критерию. Сегментация, при которой области разбиения не пересекаются,
называется \textit{тесселяцией}. Цель сегментации состоит в упрощении или
изменении представления изображения для последующей интерпретации содержимого на
изображении.
Результатом сегментации является множество сегментов, которые
покрывают все изображение. Иначе говоря, каждый пиксель отмечен меткой
некоторого класса. На сегодняшний день известно большое количество алгоритмов
сегментации изображений, использующих разные подходы. Многие из них являются
нейросетевыми. Для обучения таких алгоритмов нужно большое количество
подготовленных данных. Кроме того, при исследовании подходов к решению задачи
сегментации изображений возникает задача оценки качества выбранного алгоритма
для сравнения его с другими подходами.  Таким образом, для решения задачи
сегментации необходимо:
\begin{itemize}
    \item Найти и подготовить данные для анализа.
    \item Разработать алгоритм сегментации для полученных данных.
    \item Выбрать критерий для оценки качества алгоритма сегментации.
\end{itemize}

Сегментация изображений находит широкое применение в поиске аномалий на
медицинских изображениях, в выделении объектов на спутниковых снимках, в
системах управления дорожным движением, распознавании лиц, распознавании
отпечатков пальцев, в подготовительных работах для анализа текста на
изображении, в сельском хозяйстве, и т.д.  В данной работе задача сегментации
изображений применяется для выявления различных классов на изображении земной
поверхности.  Особенность задачи сегментации изображений земной поверхности
заключается в том, что параметры изображения зависят от многих факторов, таких
как время года, время суток, в которое был сделан снимок, угол наклона к
поверхности земли. У разных поставщиков изображения заметно отличаются.  На
практике далеко не всегда есть возможность получить изображения всех
встречающихся типов, поэтому возникает задача обучения алгоритма на одном типе
данных (например, на изображениях одного поставщика) таким образом, чтобы он был
применим и давал приемлимые результаты на данных других типов.
%image

\newpage
\section*{Поставленные задачи}
\begin{itemize}
    \item Подготовить базу изображений снимков земной поверхности.
    \item Разметить данные, то есть для каждого изображения создать ``маску'', где
        указан номер класса для каждой области.
    \item
        Опробовать различные подходы решения задачи сегментации.
    \item Обучить полносвязную конволюционную сеть для семантической
        сегментации.
    \item Сравнить различные методы решения проблемы вариативности данных.
    \item Научиться использовать источник OpenStreetMap для получения
        размеченных данных.
    \item Исследовать применимость данных OpenStreetMap для обучения алгоритмов
        сегментации земной поверхности.
\end{itemize}
